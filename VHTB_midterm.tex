\documentclass[a4paper,12pt]{article}
\usepackage[utf8]{vietnam}
\usepackage{amsmath, amssymb, amsthm}
\usepackage{geometry}
\usepackage{algorithm}
\usepackage{algpseudocode}
\geometry{margin=1in}

\theoremstyle{plain}
\newtheorem{theorem}{Định lý}
\newtheorem{lemma}{Bổ đề}
\newtheorem{corollary}{Hệ quả}

\theoremstyle{definition}
\newtheorem{problem}{Bài toán}
\newtheorem{example}{Ví dụ}

\title{Đề Thi Giữa Kỳ Tổ Hợp \& Lý Thuyết Đồ Thị Hè 2025}
\author{Lời giải sau khi thi}
\date{6 Tháng 7, 2025}

\begin{document}

\maketitle

\section*{1: Xếp Sách Vào Ngăn (THPTQG Toán 2025)}

\subsection*{Phát biểu bài toán}
Cho \( m \in \mathbb{N}^* \) ngăn trong một giá sách, được đánh số từ 1 đến \( m \), và \( n \in \mathbb{N}^* \) quyển sách phân biệt. Xếp \( n \) quyển sách này vào \( m \) ngăn, mỗi quyển sách được xếp thẳng đứng thành một hàng ngang với quy sách quay ra ngoài ở mỗi ngăn.

Khi đã xếp xong \( n \) quyển sách, hai cách xếp được gọi là \textbf{giống nhau} nếu thỏa mãn đồng thời hai điều kiện:
\begin{itemize}
    \item[(i)] Với mỗi ngăn, số lượng sách ở ngăn đó là như nhau trong cả hai cách xếp.
    \item[(ii)] Với mỗi ngăn, thứ tự từ trái sang phải của các quyển sách là như nhau.
\end{itemize}

Đếm số cách xếp đôi một khác nhau nếu:
\begin{itemize}
    \item[(a)] Mỗi ngăn có ít nhất 1 quyển sách.
    \item[(b)] Mỗi ngăn có thể không có quyển nào.
\end{itemize}

\subsection*{Phần (a): Mỗi ngăn có ít nhất 1 quyển sách}

\subsubsection*{Phân tích bài toán}
Yêu cầu:
\begin{enumerate}
    \item Phân chia \( n \) sách phân biệt vào \( m \) ngăn phân biệt.
    \item Mỗi ngăn không rỗng.
    \item Thứ tự sách trong mỗi ngăn có ý nghĩa.
\end{enumerate}

\subsubsection*{Bước 1: áp dụng Stirling loại II}
Số Stirling loại II, ký hiệu \( S(n,k) \) hoặc \( \left\{\begin{matrix} n \\ k \end{matrix}\right\} \), là số cách phân hoạch một tập \( n \) phần tử thành \( k \) tập con không rỗng.

\begin{example}
Với \( n=4 \) sách \{A, B, C, D\} và \( m=2 \) ngăn:
\begin{itemize}
    \item \{\{A, B, C\}, \{D\}\}
    \item \{\{A, B, D\}, \{C\}\}
    \item \{\{A, C, D\}, \{B\}\}
    \item \{\{B, C, D\}, \{A\}\}
    \item \{\{A, B\}, \{C, D\}\}
    \item \{\{A, C\}, \{B, D\}\}
    \item \{\{A, D\}, \{B, C\}\}
\end{itemize}
Vậy \( S(4,2) = 7 \).
\end{example}

\subsubsection*{Bước 2: Từ phân hoạch đến phân phối vào ngăn}
Mỗi phân hoạch thành \( m \) tập con không rỗng tương ứng với:
\begin{itemize}
    \item Gán mỗi tập con vào một ngăn: \( m! \) cách (ngăn phân biệt).
    \item Trong mỗi ngăn, sắp xếp sách theo thứ tự: nếu ngăn \( i \) có \( k_i \) sách, có \( k_i! \) cách.
\end{itemize}

Số cách phân phối sách vào \( m \) ngăn:
\[
S(n,m) \times m!
\]

\subsubsection*{Bước 3: Công thức tổng quát}
Xét phân phối \( n \) sách vào \( m \) ngăn, mỗi ngăn có \( k_i \geq 1 \) sách, \( k_1 + \cdots + k_m = n \):
\[
\binom{n}{k_1,k_2,\ldots,k_m} = \frac{n!}{k_1!k_2!\cdots k_m!}
\]
Tổng số cách:
\[
\sum_{\substack{k_1+\cdots+k_m=n \\ k_i \geq 1}} \binom{n}{k_1,k_2,\ldots,k_m}
\]

\subsubsection*{Bước 4: Stirling số loại II}
Số cách phân hoạch \( n \) sách thành \( m \) tập con không rỗng là \( S(n,m) \). Gán các tập con này vào \( m \) ngăn phân biệt: \( m! \) cách. Tổng số cách:
\[
m! \times S(n,m)
\]

\subsubsection*{Chứng minh tương đương}
Số cách phân phối \( n \) sách vào \( m \) ngăn không rỗng tương ứng với số hàm toàn ánh từ tập \( n \) sách sang tập \( m \) ngăn:
\[
m! \times S(n,m)
\]
Trong mỗi ngăn, thứ tự sách được xác định bởi cách phân phối, nên không cần nhân thêm \( k_i! \).

\subsubsection*{\( S(n,m) \)}
Sử dụng công thức đệ quy:
\[
S(n,m) = S(n-1,m-1) + m \cdot S(n-1,m)
\]
\begin{itemize}
    \item Cơ sở:
    \[
    S(n,n) = 1, \quad S(n,0) = 0 \text{ (nếu } n > 0\text{)}, \quad S(0,0) = 1
    \]
    \item Bước đệ quy: Xét phần tử thứ \( n \):
    \begin{itemize}
        \item Tạo tập con riêng: \( S(n-1,m-1) \).
        \item Thêm vào một trong \( m \) tập con hiện có: \( m \cdot S(n-1,m) \).
    \end{itemize}
\end{itemize}

\begin{algorithm}
\caption{Tính \( S(n,m) \)}
\begin{algorithmic}
\Function{Stirling2}{$n, m$}
    \If{$n = m$} \Return 1
    \ElsIf{$n = 0$ \textbf{or} $m = 0$} \Return 0
    \EndIf
    \State \Return \Call{Stirling2}{$n-1,m-1$} + $m$ * \Call{Stirling2}{$n-1,m$}
\EndFunction
\end{algorithmic}
\end{algorithm}

Đáp án:
\[
\boxed{m! \times S(n,m)}
\]

\subsection*{Phần (b): Mỗi ngăn có thể không có quyển nào}

\subsubsection*{Phân tích}
Mỗi sách có \( m \) lựa chọn ngăn. Số cách:
\[
m^n
\]

\subsubsection*{Chứng minh}
Xét mỗi sách độc lập:
\begin{itemize}
    \item Sách 1: \( m \) ngăn.
    \item Sách 2: \( m \) ngăn.
    \item \ldots
    \item Sách \( n \): \( m \) ngăn.
\end{itemize}
Theo quy tắc nhân:
\[
m \times m \times \cdots \times m = m^n
\]
Trong mỗi ngăn, thứ tự sách được xác định bởi thứ tự chọn sách, nên không cần thêm hoán vị.

\subsubsection*{Thuật toán}
\begin{algorithm}
\caption{Tính số cách xếp không ràng buộc}
\begin{algorithmic}
\Function{ArrangeBooks}{$n, m$}
    \State \Return $m^n$
\EndFunction
\end{algorithmic}
\end{algorithm}

\begin{example}
Với \( n=2 \), \( m=2 \):
\begin{itemize}
    \item (A,B,∅): 2 cách (AB hoặc BA).
    \item (A,∅,B): 1 cách.
    \item (B,∅,A): 1 cách.
    \item (∅,A,B): 2 cách (AB hoặc BA).
\end{itemize}
Tổng: \( 2^2 = 4 \) cách.
\end{example}

Đáp án:
\[
\boxed{m^n}
\]

\section*{2: Đẳng Thức Vandermonde}

\subsection*{Phát biểu đẳng thức}
Chứng minh:
\[
\sum_{i=0}^m \binom{m}{i}\binom{n}{r-i} = \binom{m+n}{r}, \quad \forall m,n,r \in \mathbb{N}
\]

\subsection*{Phần (a): tổ hợp}

Xét:
\begin{itemize}
    \item \( A \): \( m \) phần tử \( \{a_1, \ldots, a_m\} \).
    \item \( B \): \( n \) phần tử \( \{b_1, \ldots, b_n\} \).
    \item \( A \cup B \): \( m+n \) phần tử.
\end{itemize}

\textbf{Cách 1: Đếm trực tiếp}
Số cách chọn \( r \) phần tử từ \( A \cup B \):
\[
\binom{m+n}{r}
\]

\textbf{Cách 2: Phân loại}
Chọn:
\begin{itemize}
    \item \( i \) phần tử từ \( A \): \( \binom{m}{i} \).
    \item \( r-i \) phần tử từ \( B \): \( \binom{n}{r-i} \).
\end{itemize}
Điều kiện: \( \max(0, r-n) \leq i \leq \min(m, r) \).

Tổng:
\[
\sum_{i=\max(0,r-n)}^{\min(m,r)} \binom{m}{i}\binom{n}{r-i}
\]
Vì \( \binom{m}{i} = 0 \) nếu \( i > m \), \( \binom{n}{r-i} = 0 \) nếu \( r-i > n \), nên:
\[
\sum_{i=0}^m \binom{m}{i}\binom{n}{r-i} = \binom{m+n}{r}
\]


\begin{algorithm}
\caption{Tính tổng Vandermonde}
\begin{algorithmic}
\Function{Vandermonde}{$m, n, r$}
    \State result $\gets 0$
    \For{$i \gets 0$ \textbf{to} $m$}
        \If{$r-i \leq n$}
            \State result $\gets$ result + \Call{Binomial}{$m,i$} * \Call{Binomial}{$n,r-i$}
        \EndIf
    \EndFor
    \State \Return result
\EndFunction
\Function{Binomial}{$n,k$}
    \If{$k < 0$ \textbf{or} $k > n$} \Return 0
    \EndIf
    \State \Return $\frac{n!}{k!(n-k)!}$
\EndFunction
\end{algorithmic}
\end{algorithm}

\subsection*{Phần (b): so sánh hệ số}

Xét:
\[
(1+x)^m \times (1+x)^n = (1+x)^{m+n}
\]

Vế trái:
\[
(1+x)^m = \sum_{i=0}^m \binom{m}{i}x^i, \quad (1+x)^n = \sum_{j=0}^n \binom{n}{j}x^j
\]
Hệ số của \( x^r \):
\[
\sum_{i+j=r} \binom{m}{i}\binom{n}{j} = \sum_{i=0}^r \binom{m}{i}\binom{n}{r-i}
\]

Vế phải:
\[
(1+x)^{m+n} = \sum_{k=0}^{m+n} \binom{m+n}{k}x^k
\]
Hệ số của \( x^r \): \( \binom{m+n}{r} \).

Kết luận:
\[
\sum_{i=0}^r \binom{m}{i}\binom{n}{r-i} = \binom{m+n}{r}
\]

Tương tự phần (a), hoặc tính trực tiếp \( \binom{m+n}{r} \).

\subsection*{Phần (c): Mở rộng cho nhiều tập}

Tổng quát:
\[
\binom{\sum_{i=1}^p n_i}{m} = \sum_{\substack{k_1+\cdots+k_p=m \\ k_i \geq 0}} \prod_{i=1}^p \binom{n_i}{k_i}
\]

\textbf{quy nạp}:
\begin{itemize}
    \item \textbf{Cơ sở} (\( p=2 \)): Đẳng thức Vandermonde cơ bản.
    \item \textbf{Bước quy nạp}: Giả sử đúng với \( p-1 \). Đặt \( N = \sum_{i=1}^{p-1} n_i \):
    \[
    \binom{N}{j} = \sum_{\substack{k_1+\cdots+k_{p-1}=j \\ k_i \geq 0}} \prod_{i=1}^{p-1} \binom{n_i}{k_i}
    \]
    Áp dụng Vandermonde:
    \[
    \binom{N+n_p}{m} = \sum_{j=0}^m \binom{N}{j}\binom{n_p}{m-j}
    \]
    Thay giả thiết:
    \[
    \sum_{j=0}^m \left( \sum_{\substack{k_1+\cdots+k_{p-1}=j \\ k_i \geq 0}} \prod_{i=1}^{p-1} \binom{n_i}{k_i} \right) \binom{n_p}{m-j}
    \]
    Đặt \( k_p = m-j \):
    \[
    \sum_{\substack{k_1+\cdots+k_p=m \\ k_i \geq 0}} \prod_{i=1}^p \binom{n_i}{k_i}
    \]
\end{itemize}

\begin{algorithm}
\caption{Tính Vandermonde tổng quát}
\begin{algorithmic}
\Function{GeneralVandermonde}{$n_1,\ldots,n_p,m$}
    \If{$p = 1$} \Return \Call{Binomial}{$n_1,m$}
    \EndIf
    \State $N \gets \sum_{i=1}^{p-1} n_i$
    \State result $\gets 0$
    \For{$j \gets 0$ \textbf{to} $m$}
        \State result $\gets$ result + \Call{GeneralVandermonde}{$\{n_1,\ldots,n_{p-1}\},j$} * \Call{Binomial}{$n_p,m-j$}
    \EndFor
    \State \Return result
\EndFunction
\end{algorithmic}
\end{algorithm}

\subsection*{Phần (d): Phương pháp tính toán}

\subsubsection*{Phương pháp 1: Trực tiếp}
\[
\binom{5}{2}\binom{3}{1} + \binom{5}{1}\binom{3}{2} + \binom{5}{0}\binom{3}{3} = 30 + 15 + 1 = 46
\]
Kiểm tra: \( \binom{8}{3} = 56 \neq 46 \) (cần kiểm tra chỉ số).

\subsubsection*{Phương pháp 2: Dynamic Programming}
\begin{verbatim}
def vandermonde(m, n, r):
    C = [[0 for _ in range(max(m,n,r)+1)] for _ in range(max(m,n,r)+1)]
    for i in range(max(m,n,r)+1):
        C[i][0] = 1
        for j in range(1, i+1):
            C[i][j] = C[i-1][j-1] + C[i-1][j]
    result = 0
    for i in range(min(m+1, r+1)):
        if r-i <= n:
            result += C[m][i] * C[n][r-i]
    return result
\end{verbatim}

\section*{3: Đẳng Thức Gậy Khúc Côn Cầu}

Chứng minh:
\[
\sum_{i=0}^n \binom{i}{r} = \sum_{i=r}^n \binom{i}{r} = \binom{n+1}{r+1}, \quad \forall r,n \in \mathbb{N}, n \geq r
\]

\subsection*{Phần (a): quy nạp}

\textbf{Cơ sở}: \( n = r \):
\[
\binom{r}{r} = 1 = \binom{r+1}{r+1}
\]

\textbf{Giả thiết}: Đúng với \( n = k \):
\[
\sum_{i=r}^k \binom{i}{r} = \binom{k+1}{r+1}
\]

\textbf{quy nạp}: Với \( n = k+1 \):
\[
\sum_{i=r}^{k+1} \binom{i}{r} = \sum_{i=r}^k \binom{i}{r} + \binom{k+1}{r}
\]
\[
= \binom{k+1}{r+1} + \binom{k+1}{r} = \binom{k+2}{r+1}
\]
(vì \( \binom{k+1}{r+1} + \binom{k+1}{r} = \binom{k+2}{r+1} \)).

\begin{algorithm}
\caption{Tính tổng Hockey-stick}
\begin{algorithmic}
\Function{HockeyStick}{$n,r$}
    \State result $\gets 0$
    \For{$i \gets r$ \textbf{to} $n$}
        \State result $\gets$ result + \Call{Binomial}{$i,r$}
    \EndFor
    \State \Return result
\EndFunction
\end{algorithmic}
\end{algorithm}

\subsection*{Phần (b): biến đổi đại số}

Sử dụng:
\[
\binom{i+1}{r+1} - \binom{i}{r+1} = \binom{i}{r}
\]
Chứng minh:
\[
\binom{i+1}{r+1} - \binom{i}{r+1} = \frac{(i+1)!}{(r+1)!(i-r)!} - \frac{i!}{(r+1)!(i-r-1)!}
\]
\[
= \frac{i!(i+1 - (i-r))}{(r+1)!(i-r)!} = \binom{i}{r}
\]

Tổng:
\[
\sum_{i=r}^n \binom{i}{r} = \sum_{i=r}^n \left( \binom{i+1}{r+1} - \binom{i}{r+1} \right)
\]
\[
= \binom{n+1}{r+1} - \binom{r}{r+1} = \binom{n+1}{r+1}
\]

\subsection*{Phần (c): tổ hợp}

\textbf{Vế phải}: \( \binom{n+1}{r+1} \) = số cách chọn \( r+1 \) phần tử từ \( \{0, 1, \ldots, n\} \).

\textbf{Phân loại}: Gọi \( M \) là phần tử lớn nhất:
\begin{itemize}
    \item \( M = i \) (\( r \leq i \leq n \)).
    \item Chọn \( r \) phần tử từ \( \{0, \ldots, i-1\} \): \( \binom{i}{r} \).
\end{itemize}
Tổng:
\[
\sum_{i=r}^n \binom{i}{r} = \binom{n+1}{r+1}
\]

\subsection*{Phần (d): hàm sinh}

Xét:
\[
f(x) = \sum_{n \geq 0} \binom{n}{r} x^n = \frac{x^r}{(1-x)^{r+1}}
\]
Tổng từ \( i=r \) đến \( n \):
\[
\sum_{i=r}^n \binom{i}{r} = [x^0 + \cdots + x^{n-r}] \frac{1}{(1-x)^{r+1}} = \binom{n+1}{r+1}
\]

\section*{4: Pascal's Rule và Tổng Quát Hóa}

\subsection*{Phần (i): Phát biểu Pascal's Rule}
\[
\binom{n}{k} = \binom{n-1}{k-1} + \binom{n-1}{k}, \quad 1 \leq k \leq n-1
\]

\subsection*{Phần (ii): Hai cách chứng minh}

\subsubsection*{Đại số}
\[
\binom{n-1}{k-1} + \binom{n-1}{k} = \frac{(n-1)!}{(k-1)!(n-k)!} + \frac{(n-1)!}{k!(n-k-1)!}
\]
\[
= \frac{(n-1)!k + (n-1)!(n-k)}{k!(n-k)!} = \binom{n}{k}
\]

\subsubsection*{Tổ hợp}
Chọn \( k \) người từ \( n \) người:
\begin{itemize}
    \item Người \( n \) được chọn: \( \binom{n-1}{k-1} \).
    \item Người \( n \) không được chọn: \( \binom{n-1}{k} \).
\end{itemize}

\subsection*{Phần (c): Generalized Pascal's Rule}

Hệ số của \( x_1^{k_1}\cdots x_m^{k_m} \) trong \( (x_1 + \cdots + x_m)^n \):
\[
c(m,n,k_1,\ldots,k_m) = \binom{n}{k_1,\ldots,k_m}
\]
Đệ quy:
\[
c(m,n,k_1,\ldots,k_m) = \sum_{i=1}^m c(m,n-1,k_1,\ldots,k_i-1,\ldots,k_m)
\]

\textbf{Chứng minh}:
\[
(x_1 + \cdots + x_m)^n = (x_1 + \cdots + x_m)(x_1 + \cdots + x_m)^{n-1}
\]
Hệ số của \( x_1^{k_1}\cdots x_m^{k_m} \) được tạo từ:
\begin{itemize}
    \item \( x_i \cdot x_1^{k_1}\cdots x_i^{k_i-1}\cdots x_m^{k_m} \).
\end{itemize}

\begin{algorithm}
\caption{Tính hệ số đa thức}
\begin{algorithmic}
\Function{Multinomial}{$n,k_1,\ldots,k_m$}
    \If{$n = 0$ \textbf{and} all $k_i = 0$} \Return 1
    \ElsIf{any $k_i < 0$ \textbf{or} $\sum k_i \neq n$} \Return 0
    \EndIf
    \State result $\gets 0$
    \For{$i \gets 1$ \textbf{to} $m$}
        \State result $\gets$ result + \Call{Multinomial}{$n-1,k_1,\ldots,k_i-1,\ldots,k_m$}
    \EndFor
    \State \Return result
\EndFunction
\end{algorithmic}
\end{algorithm}

\subsection*{Phần (d): Phương pháp tính toán}

\subsubsection*{Trực tiếp}
\[
\binom{10}{3,3,4} = \frac{10!}{3!3!4!} = 4200
\]

\subsubsection*{Phân rã}
\[
\binom{10}{3,3,4} = \binom{10}{3}\binom{7}{3}\binom{4}{4} = 120 \times 35 \times 1 = 4200
\]

\section*{5: Bài Toán Chia Kẹo Euler}

\subsection*{Phần (a): Phát biểu định lý}
Số nghiệm nguyên không âm của:
\[
x_1 + \cdots + x_n = k
\]
là:
\[
\binom{k+n-1}{n-1}
\]

\subsection*{Phần (b): Chứng minh Stars and Bars}

Biểu diễn \( k \) sao và \( n-1 \) vạch. Số cách:
\[
\binom{k+n-1}{n-1}
\]

\subsubsection*{Thuật toán}
\begin{algorithm}
\caption{Tính số nghiệm Stars and Bars}
\begin{algorithmic}
\Function{StarsAndBars}{$k,n$}
    \State \Return \Call{Binomial}{$k+n-1,n-1$}
\EndFunction
\end{algorithmic}
\end{algorithm}

\subsection*{Phần (c): Điều kiện \( x_i \geq 1 \)}
Đặt \( y_i = x_i - 1 \):
\[
\sum y_i = k - n
\]
Số nghiệm:
\[
\binom{k-1}{n-1}
\]

\subsection*{Phần (d): Điều kiện \( x_i \geq a_i \)}
Đặt \( y_i = x_i - a_i \), \( S = \sum a_i \):
\[
\sum y_i = k - S
\]
Số nghiệm:
\[
\binom{k-S+n-1}{n-1}
\]

\subsection*{Phần (e): Điều kiện hỗn hợp}
\( x_i \geq 2 \) (\( i \leq m \)), \( x_i \geq 0 \) (\( i > m \)):
\[
\binom{k-2m+n-1}{n-1}
\]

\subsection*{Phần (f): Điều kiện giới hạn trên}
Dùng inclusion-exclusion:
\[
\text{Số nghiệm} = \binom{k+n-1}{n-1} - \sum_i |A_i| + \sum_{i<j} |A_i \cap A_j| - \cdots
\]
Với \( A_i \): tập nghiệm \( x_i > b_i \).

\subsubsection*{Thuật toán}
\begin{algorithm}
\caption{Tính số nghiệm với giới hạn trên}
\begin{algorithmic}
\Function{StarsAndBarsWithUpperBound}{$k,n,b_1,\ldots,b_n$}
    \State result $\gets$ \Call{StarsAndBars}{$k,n$}
    \For{each subset $S$ of $\{1,\ldots,n\}$}
        \State $s \gets \sum_{i \in S} (b_i + 1)$
        \State result $\gets$ result + $(-1)^{|S|}$ * \Call{StarsAndBars}{$k-s,n$}
    \EndFor
    \State \Return result
\EndFunction
\end{algorithmic}
\end{algorithm}

\section*{6: Đếm Đơn Thức Monic}

\subsection*{Phần (a): Đếm đơn thức bậc chính xác \( d \)}
\[
\binom{d+n-1}{n-1}
\]

\subsection*{Phần (b): Đếm đơn thức bậc \( \leq d \)}
\[
\sum_{k=0}^d \binom{k+n-1}{n-1} = \binom{d+n}{n}
\]

\subsubsection*{Chứng minh}
Sử dụng Hockey-stick identity:
\[
\sum_{k=r}^n \binom{k}{r} = \binom{n+1}{r+1}
\]
Thay \( r = n-1 \), \( k+n-1 \to k \):
\[
\sum_{k=0}^d \binom{k+n-1}{n-1} = \binom{d+n}{n}
\]

\section*{7: Bài Toán Lát Gạch Mở Rộng}

\subsection*{Phần (a): Tính \( f(1), \ldots, f(10) \)}
\[
f(n) = \sum_{k=1}^{\min(n,m)} f(n-k), \quad f(0) = 1
\]
Kết quả:
\[
f(1)=1, f(2)=2, f(3)=4, \ldots, f(10)=512
\]

\begin{algorithm}
\caption{Tính \( f(n) \)}
\begin{algorithmic}
\Function{Tile}{$n,m$}
    \If{$n = 0$} \Return 1
    \EndIf
    \State result $\gets 0$
    \For{$k \gets 1$ \textbf{to} $\min(n,m)$}
        \State result $\gets$ result + \Call{Tile}{$n-k,m$}
    \EndFor
    \State \Return result
\EndFunction
\end{algorithmic}
\end{algorithm}

\subsection*{Phần (b): Chứng minh \( f(n) = 2^{n-1} \)}
\[
f(n) = f(n-1) + \cdots + f(0)
\]
\[
f(n) - f(n-1) = f(n-1) \implies f(n) = 2f(n-1)
\]
Với \( f(1) = 1 \):
\[
f(n) = 2^{n-1}
\]

\subsection*{Phần (c): Chứng minh bằng bijection}
Mỗi cách lát tương ứng với dãy nhị phân độ dài \( n-1 \):
\begin{itemize}
    \item Bit 0: Vị trí \( i \) và \( i+1 \) thuộc cùng mảnh.
    \item Bit 1: Ranh giới giữa \( i \) và \( i+1 \).
\end{itemize}
Số dãy: \( 2^{n-1} \).

\subsection*{Phần (d): Lát hình chữ nhật \( m \times n \)}
\[
f(2,2) = 4
\]
(Do các cách lát: (1,1)|(1,1), (1,1)|(2), (2)|(1,1), (2)|(2)).

\subsection*{Phần (e): Công thức truy hồi}
Xét cột đầu tiên, công thức phức tạp, cần phân tích thêm.

\subsection*{Phần (f): Liên hệ với đồ thị}
Liên quan đến perfect matching trong grid graph.

\section*{8: Số Stirling}

\subsection*{Phần (a): Stirling số loại II}
\[
S(n,n-2) = \binom{n}{3} + \frac{1}{2}\binom{n}{2}\binom{n-2}{2}
\]

\textbf{Chứng minh}:
\begin{itemize}
    \item Một tập 3 phần tử: \( \binom{n}{3} \).
    \item Hai tập 2 phần tử: \( \frac{1}{2}\binom{n}{2}\binom{n-2}{2} \).
\end{itemize}

\subsection*{Phần (b): Stirling số loại I}
\[
s(n,n-2) = 2\binom{n}{3} + \frac{1}{2}\binom{n}{2}\binom{n-2}{2}
\]

\textbf{Chứng minh}:
\begin{itemize}
    \item Một chu trình độ dài 3: \( \binom{n}{3} \times 2 \).
    \item Hai chu trình độ dài 2: \( \frac{1}{2}\binom{n}{2}\binom{n-2}{2} \).
\end{itemize}

\section*{9: Euler's Theorem và Havel-Hakimi}

\subsection*{Phần (a): Phát biểu các định lý}

\begin{theorem}[Euler's Handshaking Lemma]
\[
\sum_{v \in V} \deg(v) = 2|E|
\]
\end{theorem}

\textbf{Hệ quả}: Số đỉnh bậc lẻ chẵn.

\textbf{Thuật toán Havel-Hakimi}:
\begin{algorithm}
\caption{Havel-Hakimi}
\begin{algorithmic}
\Function{HavelHakimi}{$d_1,\ldots,d_n$}
    \While{not all $d_i = 0$}
        \State Sort $d_i$ descending
        \If{$d_1 > n-1$ \textbf{or} any $d_i < 0$} \Return False
        \EndIf
        \For{$i \gets 2$ \textbf{to} $d_1+1$}
            \State $d_i \gets d_i - 1$
        \EndFor
        \State Remove $d_1$
        \State $n \gets n-1$
    \EndWhile
    \State \Return True
\EndFunction
\end{algorithmic}
\end{algorithm}

\subsection*{Phần (b): Áp dụng cho dãy \( 9,9,9,9,9,9,9,8,8,8 \)}
Tổng: \( 87 \) (lẻ) \(\Rightarrow\) không graphical.

\section*{10: Đồ Thị Đặc Biệt}

\subsection*{Phần (a): Tính chất chia hết}

Xét biểu thức:
\[
\frac{n(n-1)}{2}
\]
Đây là số cạnh của đồ thị đầy đủ \( K_n \), và cần chứng minh nó luôn là số nguyên.

\textbf{Chứng minh}:
\[
\frac{n(n-1)}{2}
\]
\begin{itemize}
    \item Nếu \( n \) chẵn, \( n = 2k \):
    \[
    \frac{2k(2k-1)}{2} = k(2k-1)
    \]
    Đây là tích của một số chẵn và một số lẻ, nên là số nguyên.
    \item Nếu \( n \) lẻ, \( n = 2k+1 \):
    \[
    \frac{(2k+1)2k}{2} = (2k+1)k
    \]
    Đây là tích của một số lẻ và một số chẵn, nên là số nguyên.
\end{itemize}

\textbf{Khi nào \( n \) chia hết cho \( \frac{n(n-1)}{2} \)?}
\[
\frac{n(n-1)}{2} \div n = \frac{n-1}{2}
\]
Kết quả là nguyên nếu \( n-1 \) chia hết cho 2, tức là \( n \) lẻ.

\begin{example}
\begin{itemize}
    \item \( n=3 \): \( \frac{3 \cdot 2}{2} = 3 \), \( 3 \div 3 = 1 \) (nguyên).
    \item \( n=4 \): \( \frac{4 \cdot 3}{2} = 6 \), \( 6 \div 4 = 1.5 \) (không nguyên).
    \item \( n=5 \): \( \frac{5 \cdot 4}{2} = 10 \), \( 10 \div 5 = 2 \) (nguyên).
\end{itemize}
\end{example}

\textbf{Thuật toán kiểm tra chia hết}:
\begin{algorithm}
\caption{Kiểm tra \( n \) chia hết \( \frac{n(n-1)}{2} \)}
\begin{algorithmic}
\Function{IsDivisible}{$n$}
    \State $s \gets n \cdot (n-1) / 2$
    \If{$s \mod n = 0$} \Return True
    \Else \Return False
    \EndIf
\EndFunction
\end{algorithmic}
\end{algorithm}

Đáp án:
\[
\boxed{\frac{n(n-1)}{2} \text{ luôn nguyên, chia hết cho } n \text{ khi } n \text{ lẻ}}
\]

\subsection*{Phần (b): Path Graph \( P_n \)}

\textbf{Định nghĩa}: Đồ thị đường \( P_n \) có \( n \) đỉnh, với các cạnh nối các đỉnh liên tiếp.

\textbf{Cấu trúc}:
\begin{itemize}
    \item Đỉnh: \( V = \{v_1, v_2, \ldots, v_n\} \).
    \item Cạnh: \( E = \{(v_i, v_{i+1}) : 1 \leq i \leq n-1\} \).
\end{itemize}

\textbf{Tính chất}:
\begin{itemize}
    \item Số đỉnh: \( n \).
    \item Số cạnh: \( |E| = n-1 \).
    \item Dãy bậc:
    \begin{itemize}
        \item \( v_1, v_n \): bậc 1 (đỉnh đầu và cuối).
        \item \( v_2, \ldots, v_{n-1} \): bậc 2 (đỉnh giữa).
    \end{itemize}
    Dãy bậc: \( (1, 2, \ldots, 2, 1) \).
\end{itemize}

\textbf{Chứng minh dãy bậc}:
\begin{itemize}
    \item Đỉnh \( v_1 \): Chỉ nối với \( v_2 \), bậc = 1.
    \item Đỉnh \( v_i \) (\( 2 \leq i \leq n-1 \)): Nối với \( v_{i-1} \) và \( v_{i+1} \), bậc = 2.
    \item Đỉnh \( v_n \): Chỉ nối với \( v_{n-1} \), bậc = 1.
\end{itemize}

\textbf{Kiểm tra bằng Handshaking Lemma}:
\[
\sum \deg(v_i) = 1 + (n-2) \cdot 2 + 1 = 2n - 2 = 2 \cdot (n-1)
\]

\textbf{Thuật toán xây dựng \( P_n \)}:
\begin{algorithm}
\caption{Xây dựng đồ thị \( P_n \)}
\begin{algorithmic}
\Function{BuildPathGraph}{$n$}
    \State $V \gets \{v_1, v_2, \ldots, v_n\}$
    \State $E \gets \emptyset$
    \For{$i \gets 1$ \textbf{to} $n-1$}
        \State $E \gets E \cup \{(v_i, v_{i+1})\}$
    \EndFor
    \State \Return $(V, E)$
\EndFunction
\end{algorithmic}
\end{algorithm}

\begin{example}
Cho \( n=5 \):
\begin{itemize}
    \item Cạnh: \( (v_1, v_2), (v_2, v_3), (v_3, v_4), (v_4, v_5) \).
    \item Dãy bậc: \( (1, 2, 2, 2, 1) \).
    \item Số cạnh: \( 5-1 = 4 \).
\end{itemize}
\end{example}

Đáp án:
\[
\boxed{\text{Dãy bậc: } (1, 2, \ldots, 2, 1), \ |E| = n-1}
\]

\subsection*{Phần (c): Cycle Graph \( C_n \)}

\textbf{Định nghĩa}: Đồ thị chu trình \( C_n \) có \( n \) đỉnh, với các cạnh tạo thành một chu trình khép kín.

\textbf{Cấu trúc}:
\begin{itemize}
    \item Đỉnh: \( V = \{v_1, v_2, \ldots, v_n\} \).
    \item Cạnh: \( E = \{(v_i, v_{i+1}) : 1 \leq i \leq n-1\} \cup \{(v_n, v_1)\} \).
\end{itemize}

\textbf{Tính chất}:
\begin{itemize}
    \item Số đỉnh: \( n \).
    \item Số cạnh: \( |E| = n \).
    \item Dãy bậc: Mỗi đỉnh có bậc 2 (nối với hai đỉnh lân cận).
    \item Là đồ thị 2-regular.
\end{itemize}

\textbf{Chứng minh dãy bậc}:
Mỗi đỉnh \( v_i \):
\begin{itemize}
    \item Nối với \( v_{i-1} \) (hoặc \( v_n \) nếu \( i=1 \)).
    \item Nối với \( v_{i+1} \) (hoặc \( v_1 \) nếu \( i=n \)).
\end{itemize}
Tổng bậc:
\[
n \cdot 2 = 2n = 2 \cdot |E| = 2n
\]

\textbf{Thuật toán xây dựng \( C_n \)}:
\begin{algorithm}
\caption{Xây dựng đồ thị \( C_n \)}
\begin{algorithmic}
\Function{BuildCycleGraph}{$n$}
    \State $V \gets \{v_1, v_2, \ldots, v_n\}$
    \State $E \gets \emptyset$
    \For{$i \gets 1$ \textbf{to} $n-1$}
        \State $E \gets E \cup \{(v_i, v_{i+1})\}$
    \EndFor
    \State $E \gets E \cup \{(v_n, v_1)\}$
    \State \Return $(V, E)$
\EndFunction
\end{algorithmic}
\end{algorithm}

\begin{example}
Cho \( n=5 \):
\begin{itemize}
    \item Cạnh: \( (v_1, v_2), (v_2, v_3), (v_3, v_4), (v_4, v_5), (v_5, v_1) \).
    \item Dãy bậc: \( (2, 2, 2, 2, 2) \).
    \item Số cạnh: \( 5 \).
\end{itemize}
\end{example}

Đáp án:
\[
\boxed{\text{Dãy bậc: } (2, 2, \ldots, 2), \ |E| = n}
\]

\subsection*{Phần (d): Wheel Graph \( W_n \)}

\textbf{Định nghĩa}: Đồ thị bánh xe \( W_n = C_n + K_1 \), gồm chu trình \( C_n \) và một đỉnh trung tâm nối với tất cả đỉnh trên chu trình.

\textbf{Cấu trúc}:
\begin{itemize}
    \item Đỉnh: \( V = \{c, v_1, v_2, \ldots, v_n\} \), với \( c \) là đỉnh trung tâm.
    \item Cạnh: \( E = \{(v_i, v_{i+1}) : 1 \leq i \leq n-1\} \cup \{(v_n, v_1)\} \cup \{(c, v_i) : 1 \leq i \leq n\} \).
\end{itemize}

\textbf{Tính chất}:
\begin{itemize}
    \item Số đỉnh: \( n+1 \).
    \item Số cạnh: \( n \) (chu trình) + \( n \) (từ tâm) = \( 2n \).
    \item Dãy bậc:
    \begin{itemize}
        \item Đỉnh \( c \): Nối với \( n \) đỉnh \( v_i \), bậc = \( n \).
        \item Đỉnh \( v_i \): Nối với \( v_{i-1}, v_{i+1} \) (hoặc \( v_n, v_1 \)) và \( c \), bậc = 3.
    \end{itemize}
    Dãy bậc: \( (n, 3, 3, \ldots, 3) \).
\end{itemize}

\textbf{Chứng minh dãy bậc}:
\begin{itemize}
    \item Đỉnh \( c \): Có cạnh \( (c, v_i) \) cho \( i=1,\ldots,n \), bậc = \( n \).
    \item Đỉnh \( v_i \): Có cạnh \( (v_i, v_{i-1}), (v_i, v_{i+1}), (v_i, c) \), bậc = 3.
\end{itemize}
Tổng bậc:
\[
n + n \cdot 3 = n + 3n = 4n = 2 \cdot 2n
\]

\textbf{Thuật toán xây dựng \( W_n \)}:
\begin{algorithm}
\caption{Xây dựng đồ thị \( W_n \)}
\begin{algorithmic}
\Function{BuildWheelGraph}{$n$}
    \State $V \gets \{c, v_1, v_2, \ldots, v_n\}$
    \State $E \gets \emptyset$
    \For{$i \gets 1$ \textbf{to} $n-1$}
        \State $E \gets E \cup \{(v_i, v_{i+1})\}$
    \EndFor
    \State $E \gets E \cup \{(v_n, v_1)\}$
    \For{$i \gets 1$ \textbf{to} $n$}
        \State $E \gets E \cup \{(c, v_i)\}$
    \EndFor
    \State \Return $(V, E)$
\EndFunction
\end{algorithmic}
\end{algorithm}

\begin{example}
Cho \( n=5 \):
\begin{itemize}
    \item Cạnh: \( (v_1, v_2), (v_2, v_3), (v_3, v_4), (v_4, v_5), (v_5, v_1), (c, v_1), \ldots, (c, v_5) \).
    \item Dãy bậc: \( (5, 3, 3, 3, 3, 3) \).
    \item Số cạnh: \( 2 \cdot 5 = 10 \).
\end{itemize}
\end{example}

Đáp án:
\[
\boxed{\text{Dãy bậc: } (n, 3, \ldots, 3), \ |E| = 2n}
\]

\subsection*{Phần (e): Regular Graph}

\textbf{Định nghĩa}: Đồ thị \( k \)-regular là đồ thị mà mọi đỉnh có bậc \( k \).

\textbf{Điều kiện tồn tại}: Đồ thị \( k \)-regular với \( n \) đỉnh tồn tại nếu và chỉ nếu:
\[
kn \text{ chẵn}
\]

\textbf{Chứng minh}:
\begin{itemize}
    \item Theo Handshaking Lemma:
    \[
    \sum \deg(v_i) = k \cdot n = 2|E|
    \]
    Vì \( 2|E| \) chẵn, nên \( k \cdot n \) phải chẵn.
    \item Ngược lại, nếu \( kn \) chẵn, có thể xây dựng đồ thị:
    \begin{itemize}
        \item \( k=0 \): Đồ thị rỗng.
        \item \( k=1 \): Perfect matching (\( n \) chẵn).
        \item \( k=2 \): Liên hợp các chu trình.
        \item \( k=n-1 \): Đồ thị đầy đủ \( K_n \).
    \end{itemize}
\end{itemize}

\textbf{Ví dụ}:
\begin{itemize}
    \item \( k=0 \): Đồ thị không cạnh.
    \item \( k=1 \): \( n \) chẵn, ví dụ \( K_2 \) (2 đỉnh, 1 cạnh).
    \item \( k=2 \): \( C_n \) hoặc liên hợp các chu trình.
    \item \( k=3 \): Đồ thị 3-regular, ví dụ đồ thị Petersen.
\end{itemize}

\textbf{Thuật toán kiểm tra tồn tại}:
\begin{algorithm}
\caption{Kiểm tra đồ thị \( k \)-regular}
\begin{algorithmic}
\Function{IsRegularPossible}{$n,k$}
    \If{$k \cdot n \mod 2 = 0$ \textbf{and} $k \leq n-1$} \Return True
    \Else \Return False
    \EndIf
\EndFunction
\end{algorithmic}
\end{algorithm}

Đáp án:
\[
\boxed{\text{Tồn tại nếu } kn \text{ chẵn và } k \leq n-1}
\]

\subsection*{Phần (f): Complete Bipartite Graph \( K_{m,n} \)}

\textbf{Định nghĩa}: Đồ thị hai phía đầy đủ \( K_{m,n} \) có hai tập đỉnh rời nhau, với mọi đỉnh từ tập này nối với mọi đỉnh từ tập kia.

\textbf{Cấu trúc}:
\begin{itemize}
    \item Đỉnh: \( V = A \cup B \), \( |A| = m \), \( |B| = n \), \( A \cap B = \emptyset \).
    \item Cạnh: \( E = \{(a, b) : a \in A, b \in B\} \).
\end{itemize}

\textbf{Tính chất}:
\begin{itemize}
    \item Số đỉnh: \( m+n \).
    \item Số cạnh: \( |E| = m \cdot n \).
    \item Dãy bậc:
    \begin{itemize}
        \item Mỗi đỉnh trong \( A \): bậc \( n \).
        \item Mỗi đỉnh trong \( B \): bậc \( m \).
    \end{itemize}
    Dãy bậc: \( (n, n, \ldots, n, m, m, \ldots, m) \) (\( m \) lần \( n \), \( n \) lần \( m \)).
\end{itemize}

\textbf{Chứng minh}:
\begin{itemize}
    \item Mỗi đỉnh \( a \in A \): Nối với \( n \) đỉnh trong \( B \).
    \item Mỗi đỉnh \( b \in B \): Nối với \( m \) đỉnh trong \( A \).
    \item Số cạnh:
    \[
    |E| = m \cdot n
    \]
    Tổng bậc:
    \[
    m \cdot n + n \cdot m = 2mn = 2 \cdot |E|
    \]
\end{itemize}

\textbf{Thuật toán xây dựng \( K_{m,n} \)}:
\begin{algorithm}
\caption{Xây dựng đồ thị \( K_{m,n} \)}
\begin{algorithmic}
\Function{BuildCompleteBipartite}{$m,n$}
    \State $A \gets \{a_1, \ldots, a_m\}$
    \State $B \gets \{b_1, \ldots, b_n\}$
    \State $V \gets A \cup B$
    \State $E \gets \emptyset$
    \For{each $a \in A$}
        \For{each $b \in B$}
            \State $E \gets E \cup \{(a, b)\}$
        \EndFor
    \EndFor
    \State \Return $(V, E)$
\EndFunction
\end{algorithmic}
\end{algorithm}

\begin{example}
Cho \( K_{3,2} \):
\begin{itemize}
    \item Đỉnh: \( A = \{a_1, a_2, a_3\}, B = \{b_1, b_2\} \).
    \item Cạnh: \( (a_1, b_1), (a_1, b_2), (a_2, b_1), (a_2, b_2), (a_3, b_1), (a_3, b_2) \).
    \item Dãy bậc: \( (2, 2, 2, 3, 3) \).
    \item Số cạnh: \( 3 \cdot 2 = 6 \).
\end{itemize}
\end{example}

Đáp án:
\[
\boxed{\text{Dãy bậc: } (n, \ldots, n, m, \ldots, m), \ |E| = mn}
\]

\end{document}